\documentclass{beamer}
%\usetheme{PaloAlto}
%\usetheme{Berlin}
\usetheme{Ilmenau}
\usecolortheme{seahorse}

%\usepackage[utf8]{inputenc}
%\usepackage{default}
%\usepackage[italian]{babel}

%\usepackage{titleref}
%\usepackage{zref-titleref}

\usepackage{amsmath}
\usepackage{amssymb}
\usepackage{amsthm}
\usepackage{xfrac}
\usepackage[all]{xy}
\usepackage{mathtools}
\usepackage{graphicx}
%\usepackage{fullpage}
\usepackage{hyperref}
\usepackage[utf8x]{inputenc}
\usepackage[italian]{babel}
\usepackage{mathtools}
\usepackage{colortbl}
\usepackage{multirow}
\usepackage{algorithm}
\usepackage[noend]{algpseudocode}


%\usepackage{pdftricks}
%\begin{psinputs}
%   \usepackage[pdf]{pstricks}
 %  \usepackage{multido}
%\end{psinputs}

\usepackage{ulem}

\setlength{\parindent}{0in}

\newcounter{counter1}

\theoremstyle{plain}
\newtheorem{myteo}[counter1]{Teorema}
\newtheorem{mylem}[counter1]{Lemma}
\newtheorem{mypro}[counter1]{Proposizione}
\newtheorem{mycor}[counter1]{Corollario}
\newtheorem*{myteo*}{Teorema}
\newtheorem*{mylem*}{Lemma}
\newtheorem*{mypro*}{Proposizione}
\newtheorem*{mycor*}{Corollario}

\theoremstyle{definition}
\newtheorem{mydef}[counter1]{Definizione}
\newtheorem{myes}[counter1]{Esempio}
\newtheorem{myex}[counter1]{Esercizio}
\newtheorem*{mydef*}{Definizione}
\newtheorem*{myes*}{Esempio}
\newtheorem*{myex*}{Esercizio}

\theoremstyle{remark}
\newtheorem{mynot}[counter1]{Nota}
\newtheorem{myoss}[counter1]{Osservazione}
\newtheorem*{mynot*}{Nota}
\newtheorem*{myoss*}{Osservazione}

\newcommand{\obar}[1]{\overline{#1}}
\newcommand{\ubar}[1]{\underline{#1}}

\newcommand{\set}[1]{\left\{#1\right\}}
\newcommand{\pa}[1]{\left(#1\right)}
\newcommand{\ang}[1]{\left<#1\right>}
\newcommand{\bra}[1]{\left[#1\right]}
\newcommand{\abs}[1]{\left|#1\right|}
\newcommand{\norm}[1]{\left\|#1\right\|}
\newcommand{\ceil}[1]{\left\lceil#1\right\rceil}
\newcommand{\floor}[1]{\left\lfloor#1\right\rfloor}

\newcommand{\pfrac}[2]{\pa{\frac{#1}{#2}}}
\newcommand{\bfrac}[2]{\bra{\frac{#1}{#2}}}
\newcommand{\psfrac}[2]{\pa{\sfrac{#1}{#2}}}
\newcommand{\bsfrac}[2]{\bra{\sfrac{#1}{#2}}}

\newcommand{\der}[2]{\frac{\partial #1}{\partial #2}}
\newcommand{\pder}[2]{\pfrac{\partial #1}{\partial #2}}
\newcommand{\sder}[2]{\sfrac{\partial #1}{\partial #2}}
\newcommand{\psder}[2]{\psfrac{\partial #1}{\partial #2}}

\newcommand{\intl}{\int \limits}

\DeclareMathOperator{\de}{d}
\DeclareMathOperator{\id}{Id}
\DeclareMathOperator{\len}{len}

\DeclareMathOperator{\gl}{GL}
\DeclareMathOperator{\aff}{Aff}
\DeclareMathOperator{\isom}{Isom}

\DeclareMathOperator{\im}{Im}

\DeclareMathOperator{\neigh}{neigh}



\begin{document}


\title[Oracoli per cammini su grafi]{Cammini minimi su grafi:\\
  oracoli per distanze e cammini}
%\subtitle{}
%\author{Enrico Polesel}
%\institute[Scuola Normale Superiore]{Scuola Normale Superiore}
\date{27 aprile 2015}

\author[Enrico Polesel]{\begin{tabular}{r@{ }l}
Autore: &  Enrico Polesel \\ 
Relatore: & Prof. Roberto Grossi
\end{tabular}
}



\begin{frame}[plain]
  \titlepage
\end{frame}

\begin{frame}[plain]
 \frametitle{Indice}
 \tableofcontents
\end{frame}


%\AtBeginSection[]
%{
%  \begin{frame}{\secname}
%    \tableofcontents[currentsection]
%  \end{frame}
%}


\AtBeginSubsection[]
{
  \begin{frame}[plain]{\secname $\rightarrow$ \subsecname}
    \tableofcontents[currentsubsection]
  \end{frame}
}


\section{Grafi e cammini minimi}

\subsection{Grafi e cammini}

\begin{frame}{Grafi}
  \begin{columns}
    \begin{column}{0.40\textwidth}
      \includegraphics[width=\textwidth]{directgraph}
    \end{column}
    \begin{column}{0.58\textwidth}
      \begin{itemize}
      \item<1-> $G = (V,E)$ grafo
      \item<1-> $\abs{V} = N$, $\abs{E} = M$
      \item<1-> $V$ nodi
      \item<1-> $E \subseteq V\times V$ archi
      \item<2-> gli archi possono essere orientati
      \item<2-> gli archi possono avere un peso $w: E \to \mathbb{R}$ (nel
        nostro caso la lunghezza)
      \end{itemize}
    \end{column}
  \end{columns}
\end{frame}

\begin{frame}{Cammini}
  \begin{columns}
    \begin{column}{0.40\textwidth}
      \includegraphics[width=\textwidth]{directpath}
    \end{column}
    \begin{column}{0.58\textwidth}
      \begin{itemize}
      \item<1-> Cammino: $p = \pa{ 0,7,2, 4,6}$
      \item<1-> Lunghezza: $w(p) = 7+4+10+0 = 21$
      \item<2-> $P(u,v) = \set{\text{cammini tra }u\text{ e }v}$
      \item<2-> Distanza: $\delta (u,v) = \min \limits _{p\in P(u,v)}
        w(p)$
      \item<3-> $\pa{V,\delta}$ \`e una pseudometrica
      \item<3-> I cammini minimi sono ``quasi'' aciclici
%      \item<3-> Sottocammini di cammini minimi sono minimi
      \end{itemize}
    \end{column}
  \end{columns}
\end{frame}

\begin{frame}{Ipotesi di lavoro}
  Noi useremo le seguenti ipotesi:
  \begin{itemize}
  \item grafi non pesati (ogni arco ha lunghezza $1$);
  \item grafi non diretti.
  \end{itemize}
\end{frame}

\subsection{Cammini minimi fra due nodi}

\begin{frame}{Grafo (aciclico diretto) dei cammini minimi}
  Scegliamo $s,t\in V$ e consideriamo tutti i cammini minimi fra
  loro.

  Consideriamo il grafo $G_{st}$ ottenuto considerando tutti i vertici
  e gli archi (orientati secondo il verso di percorrenza) che
  compaiono in tali cammini


    \begin{overprint}
      \onslide<1>
      \begin{itemize}
      \item
        $u \in G_{st} \Rightarrow \delta\pa{s,u} + \delta\pa{u,t} =
        \delta\pa{ s,t}$
      \end{itemize}
        \onslide<2->
        \begin{itemize}
        \item
          $u \in G_{st} \Leftrightarrow \delta\pa{s,u} +
          \delta\pa{u,t} = \delta\pa{ s,t}$
        \end{itemize}
    \end{overprint}
    \pause \pause
  \begin{itemize}
  \item
    $(u,u') \in E_{st} \Leftrightarrow \delta\pa{s,u'} =
    \delta\pa{s,u} + w(u,u') = \delta\pa{s,u} + 1$
  \item ogni cammino in $G_{st}$ \`e minimo
  \end{itemize}

  \pause
  \begin{mypro}
    Il grafo $G_{st}$ dei cammini minimi \`e un DAG (grafo aciclico diretto)
  \end{mypro}
%  Scegliendo opportunamente un cammino per ogni vertice otteniamo un
%  albero.
\end{frame}

\begin{frame}{Costruzione di $G_{st}$}
    \begin{columns}
    \begin{column}{0.40\textwidth}
      \includegraphics[width=\textwidth]{BFS}
    \end{column}
    \begin{column}{0.58\textwidth}
      Fissata una sorgente $s$ possiamo effettare una visita in
      ampiezza otteniamo il DAG dei cammini minimi in $O\pa{M}$
      operazioni.

      Da questo estraiamo $G_{st}$ al variare di $t$.
    \end{column}
  \end{columns}
\end{frame}

\begin{frame}{Costruzione dei cammini minimi}
  Definiamo, in modo induttivo, un modo per costruire i cammini minimi
  tra due nodi $s,t$ 
  \vfill

  Se $\delta(s,t) =0$ allora $s=t$ e c'\`e
  solo il cammino $(s)$.
  \vfill
  
  Supponiamo di saper costruire tutti i cammini minimi tra $s$ e $u$
  con $\delta(s,u) = k-1$ e sia $t\in V$ tale che $\delta(s,t) =k$.
  
  Sia $P_{st}$ l'insieme dei vicini di $t$ che compaiono in $G_{st}$,
  allora tutti e soli i cammini minimi da $s$ a $t$ sono dati dai
  cammini (di lunghezza $k-1$) da $s$ a $u\in P_{st}$ allungati con
  $(t)$.  \vfill \pause

  \begin{block}{}
    $u \in P_{st} \Leftrightarrow u \in \neigh (t) \wedge \delta(s,u) =
    k-1$.
  \end{block}

\end{frame}

\begin{frame}{Numero di cammini minimi}
  Dati due vertici il numero di cammini minimi tra i due potrebbe
  essere anche esponenziale
  \vfill

  \begin{overprint}
    \onslide<1>  \includegraphics[width=\textwidth]{catena0}
    \onslide<2>  \includegraphics[width=\textwidth]{catena1}
    \onslide<3>  \includegraphics[width=\textwidth]{catena2}
    \onslide<4>  \includegraphics[width=\textwidth]{catena3}
  \end{overprint}
\end{frame}

\section{Oracoli per distanze}

\subsection{Oracoli esatti}

\begin{frame}{Limiti teorici in spazio}
  \begin{itemize}
  \item Esistono $2 ^\wedge \binom{ N}{2}$ grafi indiretti non pesati
    con $N$ nodi;
  \item esiste una corrispondenza biunivoca tra grafi e matrici
    distanza
  \end{itemize}
  \pause \vfill

  Un oracolo deve saper distinguere tra $2 ^\wedge {\binom{ N}{2}}$
  casi \pause
  \begin{block}{\textit{Lower bound}}
    Un oracolo per distanze deve utilizzare almeno $N^2 / 2 + o(N^2)$
    bit
  \end{block}
\end{frame}

\begin{frame}{Risultato}
  In un articolo di \textit{Ferragina, Nitto e Venturini} comparso in
  \textit{Theoretical Computer Science} nel 2010 viene presentato un
  oracolo per grafi indiretti non pesati i cui bit utilizzati sono 
  \[ \frac{1}{2} \log \pa{3} N^2 + o\pa{N^2} \]
  \vfill
  
  Quindi si discosta dal \textit{lower bound} di un fattore $\log
  \pa{3} \simeq 1.585$.
\end{frame}

\section{Oracoli per cammini minimi}

\subsection{Oracolo per cammini}

\begin{frame}{Cammini o grafo dei cammini?}
  Esistono due modi per restituire i cammini minimi:
  \begin{itemize}
  \item (esplicito) ritornando la lista di tutti i cammini minimi;
  \item (implicito) ritornando il grafo $G_{st}$ dei cammini minimi.
  \end{itemize}
  
  L'oracolo implicito ha un output più piccolo.

  Dall'output dell'oracolo implicito si può costruire l'output
  dell'esplicito.
  \vfill 
  
  \begin{myoss}
    \`E pi\`u difficile costruire un oracolo implicito.
  \end{myoss}
\end{frame}

\begin{frame}{Riduzione ad oracoli piccoli}
  Consideriamo il problema di restituire tutti i cammini.

  Possiamo salvare:
  \begin{itemize}
  \item un oracolo per distanze $\delta(i,j)$ ($O(N^2)$ bit);
  \item una tabella booleana $A(i,j)$ dove $A(i,j)$ \`e vero se e solo
    se il numero di cammini fra $i$ e $j$ \`e minore o uguale del
    grado di $j$. ($O(N^2)$ bit).
  \end{itemize}
  \vfill
  Quando i cammini sono tanti (pi\`u del grado di $j$) possiamo
  permetterci di calcolare al volo $P_{ij}$ semplicemente testando la
  condizione $\delta\pa{s,u} = \delta\pa{s,t} -1$.
\end{frame}

\begin{frame}{Algoritmo}
  Sia $\pi (i,j)$ un oracolo che elenca i cammini minimi tra $i$ e $j$
  quando il loro numero \`e minore di $\abs{\neigh(t)}$.
  \vfill

  Vogliamo costruire un oracolo $C(i,j)$ che elenca i cammini minimi
  tra $i$ e $j$ in ogni caso.
\end{frame}

\begin{frame}
  Siano $s,t \in V$ i nodi di cui vogliamo sapere i cammini.
  \begin{algorithmic}
    \If {A(s,t)}
    \Comment {Ci sono pochi cammini}

    \Return {$\pi$(s,t)}
    \Else
    \State {P := $\emptyset$}
    \For {u $\in$ neigh(t)}
    \Comment {Calcola $P_{st}$}
    \If {$\delta$(s,u) == $\delta$(s,t)-1}
    \State {P := P $\cup$ \{u\} }
    \EndIf
    \EndFor
    \State {C(s,t) := $\emptyset$}
    \For {u $\in$ P}
    \Comment {Costruisci i cammini}
    \State {C(s,t) := C(s,t) $\cup$ (s,...,u,t) $\mid$ (s,...,u) $\in$
      C(s,u) }
    \EndFor
    \Return {C(s,t)}
    \EndIf
  \end{algorithmic}
\end{frame}

\begin{frame}{Complessit\`a}
  Si vede facilmente che lo spazio utilizzato \`e $O(N^2)$ bit pi\`u
  lo spazio usato dall'oracolo $\pi$.
  \vfill

  Il tempo utilizzato per calcolare $C(s,t)$ \`e proporzionale alla
  sua grandezza, infatti la condizione $\neg A(s,t)$ ci garantisce il
  tempo utilizzato per calcolare $P_{st}$ viene ammortizzato
  nell'output dei cammini.
\end{frame}

\begin{frame}{Esempio}
  \includegraphics[width=\textwidth]{millepieditestadiamante}
\end{frame}

\subsection{Grafi planari}

\begin{frame}{Separation theorem}
  \begin{myteo}[\textit{Separation theorem}]
    Dato un grafo planare $G = (V,E)$ \`e possibile partizionare $V$
    in tre insiemi $A,B,S$ tali che
    \begin{itemize}
    \item $A$ e $B$ hanno al pi\`u $2N/3$ vertici ciascuno
    \item $S$ ha $O\pa{ \sqrt{N}}$ vertici
    \item non esistono archi tra $A$ e $B$
    \end{itemize}
  \end{myteo}
  \vfill
  \pause
  Sottografi di grafi planari sono planari, quindi possiamo applicare
  ricorsivamente il teorema a $A\cup S$ e $B\cup S$.
\end{frame}

\begin{frame}{Dati da mantenere}
  La struttura data dal \textit{separation theorem} ci permette di
  procedere con una strategia di \textit{divide et impera}.
  \vfill \pause

  Per ogni coppia $s,t\in V$ ci salviamo i nodi separanti ``buoni'',
  cio\`e (detto $S$ un insieme che separa $s$ e $t$ dato dal
  separation theorem):
  \[ S_{s,t} = \set{ u \mid u \in S \wedge \delta\pa{s,u} +
    \delta\pa{u,t} = \delta\pa{s,t} } \]
  
  questi sono, al pi\`u, $O\pa{\sqrt{N}}$.
  \vfill \pause

  \begin{block}{Spazio utilizzato}
    \[ O \pa{N^{2.5}\log N} \text{ bit} \]
  \end{block}
\end{frame}

\begin{frame}{Un oracolo}
  Supponiamo di esserci salvati i $S_{st}$ al variare di $s,t\in V$.

  Possiamo costruire tutti i cammini minimi fra $s$ e $t$ come la
  congiunzione di tutti i cammini minimi tra $s$ e $u$ e tutti i
  cammini minimi tra $u$ e $t$.
  \vfill

  Questo \`e effettivamente un oracolo e lo spazio utilizzato \`e
  quello per salvare i $S_{st}$ che abbiamo detto essere $O\pa{
    N^{2.5} \log n}$ bit.
\end{frame}


\subsection{Oracolo per sottografo dei cammini minimi}

\begin{frame}{Un primo approccio}
  Vogliamo costruire un oracolo che dati $s,t\in V$ restituisca
  $G_{st}$.
  
  Possiamo supporre di esserci salvati, al variare di $s,t\in V$ gli
  insiemi $P_{st}$.
  \vfill
  
  Metodo ricorsivo:
  \[ G_{st} = \bra{\bigcup _{u \in P_{st}} G_{su}} \cup \pa{ P_{st}
    \cup \set{ t} , \set{ (u,t) \mid u \in P_{st} } } \]
  \pause

  Non \`e un oracolo: $G_{su}$ e $G_{sv}$ ($u\neq v$) potrebbe essere
  molto simili!
\end{frame}

\begin{frame}{Un approccio dinamico}
  \begin{algorithmic}
    \State {V$_{st}$ := \{s,t\}}
    \State {E$_{st}$ := $\emptyset$}
    \State {P := \{t\}}
    \While {P $\neq \emptyset$}
    \State {Q := $\emptyset$}
    \Comment {I prossimi predecessori buoni da visitare}
    \For {p $\in$ P}
    \State {Q := Q $\cup$ P$_{sp}$ }
    \State {V$_{st}$ := V$_{st}$ $\cup$ P$_{sp}$}
    \Comment {Inserisci i predecessori nel grafo}
    \For {u $\in$ P$_{sp}$}
    \State {E$_{st}$ := E$_{st}$ $\cup$ (u,p)}
    \EndFor
    \EndFor
    \State{P:=Q}
    \EndWhile
    \Return{G$_{st}$ := (V$_{st}$,E$_{st}$)}
  \end{algorithmic}
\end{frame}

\begin{frame}{Analisi}
  Si pu\`o vedere che questo \`e effetivamente un oracolo se gli
  insiemi vengono implementati con tabelle hash.  \vfill

  Lo spazio utilizzato \`e lo spazio utilizzato per salvare gli
  insiemi $P_{st}$.

  Possiamo vedere che:
  \[ \sum _{s,t} \abs{P_{st}} = \sum _s \sum _t \abs{P_{st}} \le \sum
  _s \sum _t \deg (t) = \sum _s 2M = 2NM \]
  quindi, ricordando che un puntatore a nodo occupa $\log N$ bit,
  abbiamo che lo spazio utilizzato \`e $O(NM\log N)$ bit.
\end{frame}

\end{document}

\section{Oracoli per cammini}

\subsection{Algoritmi facili}

\begin{frame}{Un oracolo}
  Supponiamo di conoscere la tabella delle distanze.
  \begin{mylem}
    Dati $u,v \in V$ e $u'$ vicino di $u$ esiste un cammino minimo in
    $P(u,v)$ che utilizza l'arco $(u,u')$ se e solo se $ \delta \pa{
      u,v} = w\pa{ u,u' } + \delta \pa{ u',v} $
  \end{mylem}
  \pause
  Per ogni $u,v\in V$ ci salviamo i vicini buoni di $u$ e procediamo
  in modo ricorsivo.
  \vfill \pause

  Spazio utilizzato: $O\pa{NM}$.
\end{frame}

\begin{frame}{Un non oracolo}
  Invece di precalcolarci i vicini buoni, li troviamo durante
  l'elaborazione.
  
  Spazio utilizzato: un oracolo per distanze.
  \vfill
  \pause
  \includegraphics[width=\textwidth]{millepiedi}
  
  \begin{center}
    \textbf{Non \`e un oracolo!}
  \end{center}
\end{frame}

\subsection{Grafi planari}

\begin{frame}{Separation theorem}
  \begin{myteo}[\textit{Separation theorem}]
    Dato un grafo planare $G = (V,E)$ \`e possibile partizionare $V$
    in tre insiemi $A,B,S$ tali che
    \begin{itemize}
    \item $A$ e $B$ hanno al pi\`u $2N/3$ vertici ciascuno
    \item $S$ ha $O\pa{ \sqrt{N}}$ vertici
    \item non esistono archi tra $A$ e $B$
    \end{itemize}
  \end{myteo}
  \vfill
  \pause
  Sottografi di grafi planari sono planari, quindi possiamo applicare
  ricorsivamente il teorema a $A\cup S$ e $B\cup S$.
\end{frame}

\begin{frame}{Un oracolo}
  La struttura data dal \textit{separation theorem} ci permette di
  procedere con una strategia di \textit{divide et impera}.
  \vfill \pause

  Per ogni coppia $u,v\in V$ ci salviamo i nodi separanti ``buoni''
  che sono, al pi\`u, $O\pa{\sqrt{N}}$.
  \vfill \pause

  \begin{block}{Spazio utilizzato}
    \[ O \pa{N^{2.5}} \]
  \end{block}
\end{frame}

\subsection{Miglioramenti dell'oracolo facile}

\begin{frame}{Miglioramento}
  \begin{block}{}
    Se $w$ appartiene ad un cammino minimo in $P(u,v)$ allora
    cerchiamo tutti i cammini minimi in $P(u,w)$ e
    $P(w,v)$. Concatenandoli otteniamo cammini minimi in $P\pa{u,v}$.
  \end{block}
  \vfill

  Ricorsivamente, per ogni cammino minimo in $P\pa{u,v}$ che
  scopriamo, applichiamo questo ragionamento ai suoi nodi.  
\end{frame}

\begin{frame}{Struttura dei cammini}
  Supponiamo $G$ indiretto e consideriamo il DAG $H$ dei cammini da
  $u$ a $v$ (che \`e connesso come grafo indiretto), se togliamo i due
  estremi otteniamo delle componenti connesse $H_1,...,H_k$.

  \begin{mypro}
    Per ogni componente $H_i$ ci basta conoscere un vertice $h_i \in H_i$
    per conoscere tutta la componente
  \end{mypro}
  \vfill \pause

  Invece di salvare tutti i vicini buoni salviamo un vicino buono per
  ogni componete connessa (che potrebbe contenere più di un vicino
  buono)
\end{frame}

\begin{frame}{Caso pessimo}
%    \begin{columns}
%    \begin{column}{0.6\textwidth}
      \begin{overprint}
        \onslide<1> \includegraphics[width=\textwidth]{diamante}
        \onslide<2-> \includegraphics[width=\textwidth]{diamantemultiplo}
        \begin{center}
          $\Theta \pa{N^3}$ spazio!
        \end{center}
      \end{overprint}
%    \end{column}
%    \begin{column}{0.39\textwidth}
 %    \end{column}
%  \end{columns}

\end{frame}

\subsection{Costruzione di un caso difficile}

\begin{frame}{Grafo bipartito}
  \includegraphics[width=\textwidth]{bipartito}
  \pause
  \begin{center}  
  Se $\delta\pa{a_i, a_j}=2$ allora $\mathrm{minimi}(a_i,a_j)
  \cong \mathrm{neigh}(a_i) \cap \mathrm{neigh}(a_j)$
  \end{center}
\end{frame}

\begin{frame}{Conclusioni}
  \begin{itemize}
  \item<1-> Abbiamo visto come costruire un oracolo che occupa $O\pa{NM}$
    memoria;
  \item<1-> migliorandolo siamo riusciti a trovare una struttura dei
    cammini minimi tra due vertici;
  \item<2-> \`e possibile fare di meglio? Studiare il caso difficile
    potrebbe aiutarci a capirlo.
  \end{itemize}
\end{frame}


\section*{}

\begin{frame}[plain]
 \frametitle{Indice}
 \tableofcontents
\end{frame}

\begin{frame}{Contenuti bonus}
  \begin{itemize}
  \item \hyperref[frame:prodottointeri]{Prodotto $\min +$ fra interi}
  \item \hyperref[frame:wavelettree]{Wavelet tree}
  \item \hyperref[sec:kcamm]{$k$-cammini minimi}
  \item \hyperref[sec:oracolidist]{Oracoli per distanze esatti}
  \item \hyperref[sec:approx]{Oracoli per distanze approsimati}
  \end{itemize}
\end{frame}

\begin{frame}{Prodotto $\min + $ fra interi}
\label{frame:prodottointeri}
  \begin{overprint}
    \onslide<1>
    \begin{equation*}
      \overbrace{
        \begin{pmatrix}
          a_{11}  & \cdots & a_{1N} \\
          \vdots & \ddots & \vdots \\
          a_{N1} & \cdots & a_{NN} 
        \end{pmatrix}} ^{A} *
      \overbrace{
        \begin{pmatrix}
          b_{11}  & \cdots & b_{1N} \\
          \vdots & \ddots & \vdots \\
          b_{N1} & \cdots & b_{NN} 
        \end{pmatrix}} ^{B} =
      \overbrace{
        \begin{pmatrix}
          c_{11}  & \cdots & c_{1N} \\
          \vdots & \ddots & \vdots \\
          c_{N1} & \cdots & c_{NN} 
        \end{pmatrix}} ^{C}
    \end{equation*}
    \onslide<2->
    \begin{equation*}
      \overbrace{
        \begin{pmatrix}
          x^{a_{11}}  & \cdots & x^{a_{1N}} \\
          \vdots & \ddots & \vdots \\
          x^{a_{N1}} & \cdots & x^{a_{NN}} 
        \end{pmatrix}} ^{A'} \cdot
      \overbrace{
        \begin{pmatrix}
          x^{b_{11}}  & \cdots & x^{b_{1N}} \\
          \vdots & \ddots & \vdots \\
          x^{b_{N1}} & \cdots & x^{b_{NN}} 
        \end{pmatrix}} ^{B'} =
      \overbrace{
      \begin{pmatrix}
        c'_{11}  & \cdots & c'_{1N} \\
        \vdots & \ddots & \vdots \\
        c'_{N1} & \cdots & c'_{NN} 
      \end{pmatrix}}^{C'}
    \end{equation*}
  \end{overprint}
  \begin{overprint}
    \onslide<2-> \[ c'_{ij} = \sum _k x^{a_{ik} + b_{kj}} \]
  \end{overprint}
  \[ c_{ij} = \min _k \set{ a_{ik} + b_{kj} } \]
  \begin{overprint}
    \onslide<3-> \[ c_{ij} = \mathrm{first}\pa{c'_{ij}} \]
  \end{overprint}
\end{frame}

\begin{frame}{Wavelet Tree}
  \label{frame:wavelettree}
  \begin{myteo}
    Sia $S \in\Sigma ^n$. Il Wavelet Tree costruito su $S$ occupa
    $nH_0(S) + o(n)$ bit e risponde in $\log \abs{\Sigma}$ tempo a:
    \begin{itemize}
    \item Ottieni il carattere $S[i]$
    \item $Rank_c (S,i)$: il numero di occorenze del carattere~ $c\in
      \Sigma$ nella stringa $S[0:i]$
    \item $Select _c (S,i)$: la posizione dell'$i$-esima occorrenza
      di~ $c\in \Sigma$ in~ $S$
    \end{itemize}
  \end{myteo}

  \[ \sum _{i=0} ^ {k} L[i] = \sum _{\sigma \in \Sigma} \sigma Rank
  _\sigma (S,k) \]
\end{frame}

\subsection*{$k$ cammini pi\`u brevi}
\label{sec:kcamm}

\begin{frame}
  \begin{block}{Problema}
    Dati $u,v \in V$ trovare i $k$ cammini pi\`u brevi in $P(u,v)$.
  \end{block}
  \vfill
  
  \begin{itemize}
  \item Alcuni algoritmi venono forniti in \textit{Eppstein 1994}
  \item In \textit{Aljazzar-Leue 2011}, senza peggiorare le
    prestazioni al caso pessimo, si riesce a risolvere il problema
    senza esplorare tutto il grafo
  \end{itemize}
  
\end{frame}

\subsection*{Oracoli per distanze esatti}
\label{sec:oracolidist}

\begin{frame}{Albero etichettato}
  Sia $T$ un albero di ricoprimento qualsiasi di $G$ con radice in
  $r$. Denotiamo con $f(u)$ il padre di $u$.
  \vfill
  \[ \delta\pa{ s,u} = \delta\pa{ s,r} + \sum _{v\in \pi (u)} \bra{
    \delta\pa{ s,v} - \delta\pa{ s,f(v) } } \]
  
  Per la triangolare si ha $\abs{\delta\pa{ s,v} - \delta\pa{ s,f(v)}
  } \le \abs{\delta\pa{v,f(v)}} \le 1$ \vfill
  
  \pause
  \[ l_s(v) = \delta\pa{s,v} - \delta\pa{ s, f(v) } \in \set{
    -1,0,1} \]
\end{frame}

\begin{frame}{Somme prefisse}
  \begin{columns}
    \begin{column}{0.6\textwidth}
      \includegraphics[width=\textwidth]{labeltree}
    \end{column}
    \begin{column}{0.4\textwidth}
      Associamo ad ogni etichettatura $l_s$ una sequenza $L_s$.

      Per ogni nodo $u\in V$ nella stringa compare $l(u)$ (triangolino
      verde) e $-l(u)$ (quadratino rosso).  \vfill \pause
      
      $\sum _{v\in \pi(u)} \bra{ \delta\pa{ s,v} - \delta\pa{ s,f(v) } }$
      \`e la somma prefissa fino a $r(u)$.
    \end{column}
  \end{columns}
\end{frame}

\begin{frame}{Complessità in spazio}
  Dobbiamo salvare:
  \begin{itemize}
  \item albero: $O\pa{N}$ bit
  \item distanze $\delta\pa{s,r}$ per ogni $s\in V$: $O\pa{ N \log N}$
    bit totali
  \item vettore $r$: $O\pa{N \log\pa{N}}$ bit 
  \item al variare di $s\in V$ una struttura dati associata a $L_s$:
    $2\log\pa{3} N^2 + o(N^2)$ bit totali {\color{white}
      \ref{frame:wavelettree}}
  \end{itemize}
  Asintoticamente lo spazio utilizzato è dato dalla struttura per
  $L_s$.

  \textbf{Totale:} ${2}\log\pa{3} N^2 + o(N^2)$ bit \vfill
  \pause

  Si pu\`o migliorare a 
  \begin{block}{Bit utilizzati}
    \[ \frac{1}{2}\log\pa{3} N^2 + o\pa{N^2} = 1.585 \frac{N^2}{2} +
    o\pa{N^2} \]
  \end{block}
\end{frame}


\subsection*{Oracoli per distanze approssimati}
\label{sec:approx}

\begin{frame}{Approssimazioni}
  Dover usare $\Omega \pa{N^2}$ bit potrebbe essere comunuque troppo,
  possiamo rinunciare ad avere le distanze precise in cambio di
  un'ulteriore diminuzione dello spazio occupato dall'oracolo.
  \vfill
  
  \begin{mydef}[$r$-approssimazione]
    Un algoritmo che ritorna $d(u,v)$ tale che
    \[ d\pa{u,v} \le r \delta\pa{u,v} \]
  \end{mydef}
\end{frame}

\begin{frame}{}
\setlength{\tabcolsep}{2pt}
  \begin{tabular}{c| c| c|c|c}
    \textbf{autore} & \textbf{preprocessing} & \textbf{spazio} & \textbf{tempo} & \textbf{approx} \\
    \hline 
    \multirow{3}{*}{
    \begin{tabular}{c}
      Thorup-Zwick\\
      STOC 2001
    \end{tabular}}
  & $O\pa{kMN^{1/k}}$ & $O\pa{kN^{1+1/k}}$ &
  $O\pa{k}$ & 2k-1 \\
  & $O\pa{MN^{1/2}}$ & $O\pa{N^{3/2}}$ & $O(1)$ & $3$\\
  & $O\pa{M\log N}$ & $O\pa{N\log{N}}$ & $O\pa{\log N}$  & $\log N$ \\
\hline
     \begin{tabular}{c}
       Baswana-Gaur\\
       Sen-Upadhyay\\
       LNCS 2008
     \end{tabular}
     & 
     \begin{tabular}{c}
       $O\big( M + kN^{3/2}$\\
       $N^{1/(2k)}$\\
       $N^{1/\pa{2k-2}}\big)$
     \end{tabular}
     & $O\pa{kN^{1+1/k}}$ & $O\pa{k}$ & $\pa{2k-1,2}$ 
  \end{tabular}
\setlength{\tabcolsep}{6pt}
\end{frame}

\begin{frame}{Graph spanner}
    \begin{mydef}[$t$-spanner]
    $G' = (V,E')$ con $E' \subseteq E$ tale che la distanza in $G'$
    \`e una $t$-approssimazione.
  \end{mydef}

  \begin{myteo}
    Dato un grafo $G = (V,E)$ e due numeri $t,m\ge 1$ il problema di
    determinare se esiste un $t$-spanner di $G$ con al pi\`u $m$ archi
    \`e NP-completo.
  \end{myteo}
\end{frame}


\end{document}







