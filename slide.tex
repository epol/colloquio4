\documentclass{beamer}
%\usetheme{PaloAlto}
%\usetheme{Berlin}
\usetheme{Ilmenau}
\usecolortheme{seahorse}

%\usepackage[utf8]{inputenc}
%\usepackage{default}
%\usepackage[italian]{babel}

%\usepackage{titleref}
%\usepackage{zref-titleref}

\usepackage{amsmath}
\usepackage{amssymb}
\usepackage{amsthm}
\usepackage{xfrac}
\usepackage[all]{xy}
\usepackage{mathtools}
\usepackage{graphicx}
%\usepackage{fullpage}
\usepackage{hyperref}
\usepackage[utf8x]{inputenc}
\usepackage[italian]{babel}
\usepackage{mathtools}
\usepackage{colortbl}
\usepackage{multirow}
\usepackage{algorithm}
\usepackage[noend]{algpseudocode}


%\usepackage{pdftricks}
%\begin{psinputs}
%   \usepackage[pdf]{pstricks}
 %  \usepackage{multido}
%\end{psinputs}

\usepackage{ulem}

\setlength{\parindent}{0in}

\newcounter{counter1}

\theoremstyle{plain}
\newtheorem{myteo}[counter1]{Teorema}
\newtheorem{mylem}[counter1]{Lemma}
\newtheorem{mypro}[counter1]{Proposizione}
\newtheorem{mycor}[counter1]{Corollario}
\newtheorem{mycon}[counter1]{Congettura}
\newtheorem*{myteo*}{Teorema}
\newtheorem*{mylem*}{Lemma}
\newtheorem*{mypro*}{Proposizione}
\newtheorem*{mycor*}{Corollario}
\newtheorem*{mycon*}{Congettura}

\theoremstyle{definition}
\newtheorem{mydef}[counter1]{Definizione}
\newtheorem{myes}[counter1]{Esempio}
\newtheorem{myex}[counter1]{Esercizio}
\newtheorem*{mydef*}{Definizione}
\newtheorem*{myes*}{Esempio}
\newtheorem*{myex*}{Esercizio}

\theoremstyle{remark}
\newtheorem{mynot}[counter1]{Nota}
\newtheorem{myoss}[counter1]{Osservazione}
\newtheorem*{mynot*}{Nota}
\newtheorem*{myoss*}{Osservazione}

\newcommand{\obar}[1]{\overline{#1}}
\newcommand{\ubar}[1]{\underline{#1}}

\newcommand{\set}[1]{\left\{#1\right\}}
\newcommand{\pa}[1]{\left(#1\right)}
\newcommand{\ang}[1]{\left<#1\right>}
\newcommand{\bra}[1]{\left[#1\right]}
\newcommand{\abs}[1]{\left|#1\right|}
\newcommand{\norm}[1]{\left\|#1\right\|}
\newcommand{\ceil}[1]{\left\lceil#1\right\rceil}
\newcommand{\floor}[1]{\left\lfloor#1\right\rfloor}

\newcommand{\pfrac}[2]{\pa{\frac{#1}{#2}}}
\newcommand{\bfrac}[2]{\bra{\frac{#1}{#2}}}
\newcommand{\psfrac}[2]{\pa{\sfrac{#1}{#2}}}
\newcommand{\bsfrac}[2]{\bra{\sfrac{#1}{#2}}}

\newcommand{\der}[2]{\frac{\partial #1}{\partial #2}}
\newcommand{\pder}[2]{\pfrac{\partial #1}{\partial #2}}
\newcommand{\sder}[2]{\sfrac{\partial #1}{\partial #2}}
\newcommand{\psder}[2]{\psfrac{\partial #1}{\partial #2}}

\newcommand{\intl}{\int \limits}

\DeclareMathOperator{\de}{d}
\DeclareMathOperator{\id}{Id}
\DeclareMathOperator{\len}{len}

\DeclareMathOperator{\gl}{GL}
\DeclareMathOperator{\aff}{Aff}
\DeclareMathOperator{\isom}{Isom}

\DeclareMathOperator{\im}{Im}

\DeclareMathOperator{\neigh}{neigh}



\begin{document}


\title[Oracoli per cammini minimi su grafi]{Cammini minimi su grafi:\\
  oracoli per distanze e cammini}
%\subtitle{}
%\author{Enrico Polesel}
%\institute[Scuola Normale Superiore]{Scuola Normale Superiore}
\date{27 aprile 2015}

\author[Enrico Polesel]{\begin{tabular}{r@{ }l}
Autore: &  Enrico Polesel \\ 
Relatore: & Prof. Roberto Grossi
\end{tabular}
}



\begin{frame}[plain]
  \titlepage
\end{frame}

\begin{frame}[plain]
 \frametitle{Indice}
 \tableofcontents
\end{frame}


%\AtBeginSection[]
%{
%  \begin{frame}{\secname}
%    \tableofcontents[currentsection]
%  \end{frame}
%}


\AtBeginSubsection[]
{
  \begin{frame}[plain]{\secname $\rightarrow$ \subsecname}
    \tableofcontents[currentsubsection]
  \end{frame}
}


\section{Grafi e cammini minimi}

\subsection{Grafi e cammini}

\begin{frame}{Grafi}
  \begin{columns}
    \begin{column}{0.40\textwidth}
      \includegraphics[width=\textwidth]{directgraph}
    \end{column}
    \begin{column}{0.58\textwidth}
      \begin{itemize}
      \item<1-> $G = (V,E)$ grafo
      \item<1-> $\abs{V} = N$, $\abs{E} = M$
      \item<1-> $V$ nodi
      \item<1-> $E \subseteq V\times V$ archi
      \item<2-> gli archi possono essere orientati
      \item<2-> gli archi possono avere un peso $w: E \to \mathbb{R}$ (nel
        nostro caso la lunghezza)
      \end{itemize}
    \end{column}
  \end{columns}
\end{frame}

\begin{frame}{Cammini}
  \begin{columns}
    \begin{column}{0.40\textwidth}
      \includegraphics[width=\textwidth]{directpath}
    \end{column}
    \begin{column}{0.58\textwidth}
      \begin{itemize}
      \item<1-> Cammino: $p = \pa{ 0,7,2, 4,6}$;
      \item<1-> lunghezza: $w(p) = 7+4+10+0 = 21$;
      \item<2-> $P(u,v) = \set{\text{cammini tra }u\text{ e }v}$;
      \item<2-> "distanza": $\delta (u,v) = \min \limits _{p\in P(u,v)}
        w(p)$;
%      \item<3-> $\pa{V,\delta}$ \`e una pseudometrica
%      \item<3-> I cammini minimi sono ``quasi'' aciclici
      \item<3-> sottocammini di cammini minimi sono minimi.
      \end{itemize}
    \end{column}
  \end{columns}
\end{frame}

\begin{frame}{Ipotesi di lavoro}
  Noi useremo le seguenti ipotesi:
  \begin{itemize}
  \item grafi connessi (ci si pu\`o facilmente ricondurre a questa
    situazione);
  \item grafi non pesati (ogni arco ha lunghezza $1$);
  \item grafi non diretti.
  \end{itemize}
  \vfill
  
  Conseguenze:
  \begin{itemize}
  \item non esistono cicli di lunghezza negativa;
  \item $\forall s,t \; \delta\pa{s,t} \le N < \infty$;
  \item la distanza ci definisce una metrica.
  \end{itemize}
\end{frame}

\subsection{Cammini minimi fra due nodi}

\begin{frame}{Grafo (aciclico diretto) dei cammini minimi}
  Scegliamo $s,t\in V$ e consideriamo tutti i cammini minimi fra
  loro.

  Consideriamo il grafo $G_{st}$ ottenuto considerando tutti i vertici
  e gli archi (orientati secondo il verso di percorrenza) che
  compaiono in tali cammini


    \begin{overprint}
      \onslide<1>
      \begin{itemize}
      \item
        $u \in G_{st} \Rightarrow \delta\pa{s,u} + \delta\pa{u,t} =
        \delta\pa{ s,t}$
      \end{itemize}
        \onslide<2->
        \begin{itemize}
        \item
          $u \in G_{st} \Leftrightarrow \delta\pa{s,u} +
          \delta\pa{u,t} = \delta\pa{ s,t}$
        \end{itemize}
    \end{overprint}
    \pause \pause
  \begin{itemize}
  \item
    $(u,u') \in E_{st} \Leftrightarrow \delta\pa{s,u'} =
    \delta\pa{s,u} + w(u,u') = \delta\pa{s,u} + 1$
  \item ogni cammino in $G_{st}$ \`e minimo
  \end{itemize}

  \pause
  \begin{mypro}
    Il grafo $G_{st}$ dei cammini minimi \`e un DAG (grafo aciclico diretto)
  \end{mypro}
%  Scegliendo opportunamente un cammino per ogni vertice otteniamo un
%  albero.
\end{frame}

\begin{frame}{Costruzione di $G_{st}$}
    \begin{columns}
    \begin{column}{0.40\textwidth}
      \includegraphics[width=\textwidth]{BFS}
    \end{column}
    \begin{column}{0.58\textwidth}
      Fissata una sorgente $s$ possiamo effettuare una visita in
      ampiezza otteniamo il DAG dei cammini minimi in $O\pa{M}$
      operazioni.

      Da questo estraiamo $G_{st}$ al variare di $t$.
    \end{column}
  \end{columns}
\end{frame}

\begin{frame}{Costruzione dei cammini minimi}
  Definiamo, in modo induttivo, un modo per costruire i cammini minimi
  tra due nodi $s,t$ 
  \vfill

  Se $\delta(s,t) =0$ allora $s=t$ e c'\`e
  solo il cammino $(s)$.
  \vfill
  
  Supponiamo di saper costruire tutti i cammini minimi tra $s$ e $u$
  con $\delta(s,u) = k-1$ e sia $t\in V$ tale che $\delta(s,t) =k$.
  
  Sia $P_{st}$ l'insieme dei vicini di $t$ che compaiono in $G_{st}$,
  allora tutti e soli i cammini minimi da $s$ a $t$ sono dati dai
  cammini (di lunghezza $k-1$) da $s$ a $u\in P_{st}$ allungati con
  $(t)$.  \vfill \pause

  \begin{block}{}
    $u \in P_{st} \Leftrightarrow u \in \neigh (t) \wedge \delta(s,u) =
    k-1$.
  \end{block}
\end{frame}

\begin{frame}{Generalizzazione della costruzione}
  Scegliamo un  $M \subseteq V_{st}$.

  Al variare di $u \in M$ tutti i cammini ottenuti concatenando un
  cammino minimo da $s$ a $u$ con uno da $u$ a $t$ sono cammini minimi
  da $s$ a $t$.
  \vfill

  Se $M$ \`e scelto in modo opportuno (per esempio $M = P_{st}$)
  allora questi saranno anche i soli cammini minimi da $s$ a $t$.
\end{frame}

\begin{frame}{Numero di cammini minimi}
  Dati due vertici il numero di cammini minimi tra i due potrebbe
  essere anche esponenziale
  \vfill

  \begin{overprint}
    \onslide<1>  \includegraphics[width=\textwidth]{catena0}
    \onslide<2>  \includegraphics[width=\textwidth]{catena1}
    \onslide<3>  \includegraphics[width=\textwidth]{catena2}
    \onslide<4>  \includegraphics[width=\textwidth]{catena3}
  \end{overprint}
\end{frame}

\subsection{Oracoli per distanze esatti}

\begin{frame}{Limiti teorici in spazio}
  \begin{itemize}
  \item Esistono $2 ^\wedge \binom{ N}{2}$ grafi indiretti non pesati
    con $N$ nodi;
  \item esiste una corrispondenza biunivoca tra grafi e matrici
    distanza
  \end{itemize}
  \pause \vfill

  Un oracolo deve saper distinguere tra $2 ^\wedge {\binom{ N}{2}}$
  casi \pause
  \begin{block}{\textit{Lower bound}}
    Un oracolo per distanze deve utilizzare almeno $N^2 / 2 + o(N^2)$
    bit
  \end{block}
\end{frame}

\begin{frame}{Risultato}
  Esiste un oracolo per grafi indiretti non pesati i cui bit
  utilizzati sono:
  \[ \frac{1}{2} \log \pa{3} N^2 + o\pa{N^2}. \]
  \vfill
  
  Quindi si discosta dal \textit{lower bound} di un fattore $\log
  \pa{3} \simeq 1.585$.
\end{frame}

\section{Oracoli per cammini minimi}

\subsection{L'insieme $P_{st}$}

\begin{frame}{Salvare l'insieme}
  Possiamo salvare, al variare di $s,t\in V$ l'insieme $P_{st}$.
  \vfill
  
  \[ \sum _{s,t} \abs{ P_{st} } = \sum _s \sum _t \abs{ P_{st} } \le
  \sum _s \sum _t \abs{ \neigh (t) } = \sum _s 2M = 2MN \]
  Quindi $O(MN\log N) = O(N^3\log N)$ bit!
\end{frame}

\begin{frame}{Caso pessimo}
%    \begin{columns}
%    \begin{column}{0.6\textwidth}
      \begin{overprint}
        \onslide<1> \includegraphics[width=\textwidth]{diamante}
        \onslide<2-> \includegraphics[width=\textwidth]{diamantemultiplo}
        \begin{center}
          $\Theta \pa{N^3\log N}$ bit!
        \end{center}
      \end{overprint}
%    \end{column}
%    \begin{column}{0.39\textwidth}
 %    \end{column}
%  \end{columns}
\end{frame}

\begin{frame}{Calcolare l'insieme al volo}
  \[ P_{st} = \set{ u \in \neigh(t) \mid \delta\pa{s,t} =
    \delta\pa{s,u} + 1 } \]
  \vfill \pause
  \includegraphics[width=\textwidth]{millepiedi}
\end{frame}

\subsection{Oracolo per cammini}

\begin{frame}{Cammini o grafo dei cammini?}
  Esistono due modi per restituire i cammini minimi:
  \begin{itemize}
  \item (esplicito) ritornando la lista di tutti i cammini minimi;
  \item (implicito) ritornando il grafo $G_{st}$ dei cammini minimi.
  \end{itemize}
  
%  L'oracolo implicito ha un output più piccolo.

%  Dall'output dell'oracolo implicito si può costruire l'output
%  dell'esplicito.
  \vfill \pause
  
  \begin{myoss}
    \`E pi\`u difficile costruire un oracolo implicito.
  \end{myoss}
\end{frame}

\begin{frame}{Riduzione ad oracoli piccoli}
  Consideriamo il problema di restituire tutti i cammini.

  Se ci sono abbastanza cammini da stampare possiamo pensare di
  calcolare al volo $P_{st}$. Altrimenti dobbiamo calcolare $P_{st}$
  in un altro modo (oppure non utilizzare $P_{st}$).
  \vfill 
  
  \[ A(s,t) = \mathcal{B}\bra{\abs{mpath(s,t)} \le \abs{neigh(t)}} \]
%  Possiamo salvare:
%  \begin{itemize}
%  \item un oracolo per distanze $\delta(i,j)$ ($O(N^2)$ bit);
%  \item una tabella booleana $A(i,j)$ dove $A(i,j)$ \`e vero se e solo
%    se il numero di cammini fra $i$ e $j$ \`e minore o uguale del
%    grado di $j$. ($O(N^2)$ bit).
%  \end{itemize}
%  \vfill
%  Quando i cammini sono tanti (pi\`u del grado di $j$) possiamo
%  permetterci di calcolare al volo $P_{ij}$ semplicemente testando la
%  condizione $\delta\pa{s,u} = \delta\pa{s,t} -1$.
\end{frame}

\begin{frame}{Algoritmo}
  Sia $\pi (s,t)$ un oracolo che elenca i cammini minimi tra $s$ e $t$
  quando il loro numero \`e minore o uguale di $\abs{\neigh(t)}$.
  \vfill

  Vogliamo costruire un oracolo $C(s,t)$ che elenca i cammini minimi
  tra $s$ e $t$ in ogni caso.
\end{frame}

\begin{frame}
  Siano $s,t \in V$ i nodi di cui vogliamo sapere i cammini.
  \begin{algorithmic}
    \If {A(s,t)}
    \Comment {Ci sono pochi cammini}

    \Return {$\pi$(s,t)}
    \Else
    \State {P := $\emptyset$}
    \For {u $\in$ neigh(t)}
    \Comment {Calcola $P_{st}$}
    \If {$\delta$(s,u) == $\delta$(s,t)-1}
    \State {P := P $\cup$ \{u\} }
    \EndIf
    \EndFor
    \State {C(s,t) := $\emptyset$}
    \For {u $\in$ P}
    \Comment {Costruisci i cammini}
    \State {C(s,t) := C(s,t) $\cup$ (s,...,u,t) $\mid$ (s,...,u) $\in$
      C(s,u) }
    \EndFor
    \Return {C(s,t)}
    \EndIf
  \end{algorithmic}
\end{frame}

\begin{frame}{Complessit\`a}
  Si vede facilmente che lo spazio utilizzato \`e $O(N^2)$ bit pi\`u
  lo spazio usato dall'oracolo $\pi$.
  \vfill

  Il tempo utilizzato per calcolare $C(s,t)$ \`e proporzionale alla
  sua grandezza, infatti la condizione $\neg A(s,t)$ ci garantisce il
  tempo utilizzato per calcolare $P_{st}$ viene ammortizzato
  nell'output dei cammini.
\end{frame}

\begin{frame}{Esempio}
  \includegraphics[width=\textwidth]{millepieditestadiamante}
\end{frame}


\subsection{Oracolo per sottografo dei cammini minimi}

\begin{frame}{Un primo approccio}
  Vogliamo costruire un oracolo che dati $s,t\in V$ restituisca
  $G_{st}$.
  
  Supponiamo di saper costruire velocemente, al variare di $s,t\in V$,
  gli insiemi $P_{st}$.  \vfill
  
  Metodo ricorsivo:
  \[ G_{st} = \bra{\bigcup _{u \in P_{st}} G_{su}} \cup \pa{ P_{st}
    \cup \set{ t} , \set{ (u,t) \mid u \in P_{st} } } \]
  \pause

  Non \`e un oracolo: $G_{su}$ e $G_{sv}$ ($u\neq v$) potrebbe essere
  molto simili!
\end{frame}

\begin{frame}{Esempio}
  \includegraphics[angle=180,width=\textwidth]{millepieditestadiamante}
\end{frame}

\begin{frame}{Un approccio dinamico}
  \begin{algorithmic}
    \State {V$_{st}$ := \{s,t\}}
    \State {E$_{st}$ := $\emptyset$}
    \State {P := \{t\}}
    \While {P $\neq \emptyset$}
    \State {Q := $\emptyset$}
    \Comment {I prossimi predecessori buoni da visitare}
    \For {p $\in$ P}
    \State {Q := Q $\cup$ P$_{sp}$ }
    \State {V$_{st}$ := V$_{st}$ $\cup$ P$_{sp}$}
    \Comment {Inserisci i predecessori nel grafo}
    \For {u $\in$ P$_{sp}$}
    \State {E$_{st}$ := E$_{st}$ $\cup$ (u,p)}
    \EndFor
    \EndFor
    \State{P:=Q}
    \EndWhile
    \Return{G$_{st}$ := (V$_{st}$,E$_{st}$)}
  \end{algorithmic}
\end{frame}

\begin{frame}{Analisi}
  Si pu\`o vedere che questo \`e effettivamente un oracolo se gli
  insiemi vengono implementati con tabelle hash.  \vfill

  Lo spazio utilizzato \`e lo spazio utilizzato per salvare gli
  insiemi $P_{st}$ che sappiamo essere $O(NM\log N)$.
\end{frame}


\section{Intersezione di insiemi}

\subsection{Oracoli per intersezioni di insiemi}

\begin{frame}{Intersezione di insiemi}
  Sia $X$ un insieme ambiente e supponiamo di avere
  \[ S_1, ... , S_n \in \mathcal{P}(X) \]
  
  Siamo interessati a rispondere ad interrogazioni riguardo $S_i \cap
  S_j$.
  \vfill

  Esistono due tipi principali di interrogazioni:
  \begin{itemize}
  \item $S_i \cap S_j$ \`e non vuoto?
  \item enumerare gli elementi di $S_i \cap S_j$.
  \end{itemize}
  Evidentemente il secondo problema \`e pi\`u difficile del primo.
\end{frame}

\begin{frame}{Oracoli ``vuoto o non vuoto''}
  \begin{mycon}
    Consideriamo un oracolo che preprocessa $S_1,...,S_n$ e risponde
    alla domanda ``$S_i$ interseca $S_j$?''. Sia $\abs{X} = \log _c n$
    per $c$ sufficientemente grande. Allora tale oracolo deve occupare
    $\tilde \Omega (n^2)$ memoria.
  \end{mycon}
  \vfill
  
  Nelle ipotesi della congettura scopriamo che, a meno di fattori
  logaritmici, l'oracolo ``banale'' che salva tutte le possibili
  risposte (e occupa $O(n^2)$ bit) \`e ``ottimo'' in spazio.
\end{frame}

\subsection{Relazione con gli oracoli per cammini minimi}

\begin{frame}{Riduzione da intersezioni a cammini}
  Costruiamo un grafo bipartito $G = (V,E)$ nel seguente modo:
  \[ V = X \cup \set{ S_1, ... ,S_n} \;\; E = \set{ ( S_i, x) \mid x
    \in S_i } \] \vfill
  
  Vediamo che, se $S_i \cap S_j \neq \emptyset$, si ha che i cammini
  minimi tra $S_i$ e $S_j$ sono tutti e soli i:
  \[ \set{ ( S_i, x, S_j) \mid x \in S_i \cap S_j }\]

  \includegraphics[width=\textwidth]{setintersectiongraph}
\end{frame}

\begin{frame}{Osservazioni sulla riduzione}
  Sia $m = \sum _i \abs{S_i}$, allora il grafo costruito ha $N = n +
  \abs{X}$ nodi e $M = m$ archi.
  
  Per $\abs{X}$ piccolo ($\abs{X} = \log _c (n)$) abbiamo costruito un
  oracolo per intersezioni in $O\pa{nm\log n}$ bit che \`e lontano
  dall'attuale lower bound di $\tilde \Omega\pa{n^2}$ bit.  \vfill
  \pause

  Quindi costruire un oracolo per cammini minimi \`e pi\`u difficile
  di costruire un oracolo per enumerare l'intersezione di insiemi.
  \pause

  Eventuali lower bound per l'intersezione passano ai cammini minimi.
\end{frame}

% \begin{frame}{Riduzione da cammini a intersezione}
%   Sia $G = (V,E)$ un grafo e definiamo:
%   \[ L_s (k) = \set{ u \in V \mid \delta\pa{s,u} = k } \]
%   \pause
%   Possiamo quindi esprimere l'insieme $P_{st}$ come:
%   \[ P_{st} = L_s(\delta\pa{s,t} -1) \cap L_t(1) \]
%   \vfill \pause

%   Supponendo di avere un oracolo per distanze e un oracolo per le
%   intersezioni dei $L_s(k)$ possiamo costruire un oracolo per cammini
%   minimi.
% \end{frame}

% \begin{frame}{Osservazioni sulla riduzione}
%   Abbiamo ottenuto $n = dN$ insiemi (dove $d$ \`e il diametro del
%   grafo) la cui somma delle cardinalit\`a \`e:
%   \[ m = \sum _s \sum _k  L_s(k) = \sum _s N = N^2 \]
%   \vfill \pause

%   Quindi, supponendo di conoscere un buon oracolo per intersezioni,
%   potremmo riuscire a costruire un buon oracolo per grafi di
%   \textbf{diametro piccolo} (per esempio le reti sociali).

%   Questo comunque occuper\`a $\tilde \Omega (n^2) = \tilde \Omega
%   (d^2N^2)$ bit.
% \end{frame}


\section{Conclusioni e possibili sviluppi}

\subsection{}

\begin{frame}[plain]{\secname}
  \tableofcontents[currentsection]
\end{frame}


\begin{frame}{Sui piccoli miglioramenti}
  L'algoritmo principale che abbiamo presentato utilizza $O\pa{ NM
    \log N}$ bit di memoria. Il nostro lavoro si \`e concentrato sul
  cercare di migliorare i fattori polinomiali legati alle dimensioni
  del grafo.
  \vfill
  
  Restano aperti due possibili miglioramenti:
  \begin{itemize}
  \item migliorare i fattori logaritmici (per esempio eliminando il~
    $\log N$);
  \item sfruttare le operazioni binari su word per migliorare di
    fattori~ $w$ (lunghezza di una word, per esempio $64$ bit).
  \end{itemize}
\end{frame}

\begin{frame}{Sulle intersezioni}
  Abbiamo mostrato un'interessante collegamento con il problema
  dell'intersezione di insiemi.
  
  Questo problema si ritrova in altri contesti, per esempio nelle basi
  di dati.
  \vfill

  Restano aperti due problemi:
  \begin{itemize}
  \item dimostrare un lower bound migliore per il problema (che
    verrebbe riportato al problema dei cammini);
  \item trovare una riduzione dai cammini alle intersezioni.
  \end{itemize}
\end{frame}

\begin{frame}{Generalizzazioni}
  Abbiamo lavorato per semplicit\`a con ipotesi abbastanza stringenti
  (grafi indiretti non pesati).  \vfill
  
  Possiamo studiare:
  \begin{itemize}
  \item generalizzazioni a grafi diretti;
  \item generalizzazioni a grafi pesati.
  \end{itemize}
\end{frame}

\begin{frame}{Problemi collegati}
  Varianti di questo problema sono (gi\`a studiati per le distanze):
  \vfill
  \begin{itemize}
  \item ritornare un'approssimazione di tutti i cammini minimi (per
    esempio tutti i cammini la cui lunghezza \`e una
    $r$-approssimazione della distanza);
  \item costruire algoritmi dinamici (cio\`e con inserimento e
    rimozione di archi a tempo di esecuzione) per ritornare i cammini
    minimi.
  \end{itemize}
\end{frame}

\begin{frame}{Congetture}
  Presentiamo le seguenti congetture:
  \begin{mycon}
    Un oracolo per ritornare il grafo dei cammini minimi deve
    utilizzare $\tilde \Omega \pa{ NM}$ memoria.
  \end{mycon}
  \begin{mycon}
    Un oracolo per elencare tutti i cammini minimi deve
    utilizzare $\tilde \Omega \pa{ NM}$ memoria.
  \end{mycon}
  \begin{mycon}
    Un oracolo per elencare tutti gli elementi dell'intersezione di
    insiemi deve utilizzare $\tilde \Omega \pa{ nm}$ memoria.
  \end{mycon}
\end{frame}


\end{document}







