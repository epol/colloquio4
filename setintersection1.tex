\documentclass[a4paper,10pt]{article}

\usepackage{amsmath}
\usepackage{amssymb}
\usepackage{amsthm}
\usepackage{xfrac}
\usepackage[all]{xy}
\usepackage{graphicx}
%\usepackage{fullpage}
\usepackage{hyperref}
\usepackage[utf8x]{inputenc}
\usepackage[english]{babel}

%\setlength{\parindent}{0in}

\newcounter{counter1}

\theoremstyle{plain}
\newtheorem{myteo}[counter1]{Teorema}
\newtheorem{mylem}[counter1]{Lemma}
\newtheorem{mypro}[counter1]{Proposizione}
\newtheorem{mycor}[counter1]{Corollario}
\newtheorem*{myteo*}{Teorema}
\newtheorem*{mylem*}{Lemma}
\newtheorem*{mypro*}{Proposizione}
\newtheorem*{mycor*}{Corollario}

\theoremstyle{definition}
\newtheorem{mydef}[counter1]{Definition}
\newtheorem{myes}[counter1]{Esempio}
\newtheorem{myex}[counter1]{Esercizio}
\newtheorem*{mydef*}{Definition}
\newtheorem*{myes*}{Esempio}
\newtheorem*{myex*}{Esercizio}

\theoremstyle{remark}
\newtheorem{mynot}[counter1]{Nota}
\newtheorem{myoss}[counter1]{Osservazione}
\newtheorem*{mynot*}{Nota}
\newtheorem*{myoss*}{Osservazione}


\newcommand{\obar}[1]{\overline{#1}}
\newcommand{\ubar}[1]{\underline{#1}}

\newcommand{\set}[1]{\left\{#1\right\}}
\newcommand{\pa}[1]{\left(#1\right)}
\newcommand{\ang}[1]{\left<#1\right>}
\newcommand{\bra}[1]{\left[#1\right]}
\newcommand{\abs}[1]{\left|#1\right|}
\newcommand{\norm}[1]{\left\|#1\right\|}

\newcommand{\pfrac}[2]{\pa{\frac{#1}{#2}}}
\newcommand{\bfrac}[2]{\bra{\frac{#1}{#2}}}
\newcommand{\psfrac}[2]{\pa{\sfrac{#1}{#2}}}
\newcommand{\bsfrac}[2]{\bra{\sfrac{#1}{#2}}}

\newcommand{\der}[2]{\frac{\partial #1}{\partial #2}}
\newcommand{\pder}[2]{\pfrac{\partial #1}{\partial #2}}
\newcommand{\sder}[2]{\sfrac{\partial #1}{\partial #2}}
\newcommand{\psder}[2]{\psfrac{\partial #1}{\partial #2}}

\newcommand{\intl}{\int \limits}

\DeclareMathOperator{\de}{d}
\DeclareMathOperator{\id}{Id}
\DeclareMathOperator{\len}{len}

\DeclareMathOperator{\gl}{GL}
\DeclareMathOperator{\aff}{Aff}
\DeclareMathOperator{\isom}{Isom}

\DeclareMathOperator{\im}{Im}

\DeclareMathOperator{\neigh}{neigh}


\title{Relazione con l'intersezione di insiemi}
\author{Enrico Polesel}
\date{\today}

\begin{document}
\maketitle

\section{Intersezione di insiemi}

\subsection{Definizione}

Consideriamo $n$ insiemi $S_1, ..., S_n$ sottoinsiemi di un insieme
ambiente $X$. Siamo interessati agli insiemi $S_i \cap S_j$ al variare
di $i$ e $j$. Possiamo supporre $\forall i \; S_i \neq \emptyset$.

Quindi il nostro input avr\`a grandezza 
\[ \Theta \pa{ n + \sum _i \abs{S_i}} = \Theta \pa { \sum _i \abs{S_i}
} = \Theta (m) \]
dove abbiamo definito $m = \sum _i \abs{S_i}$.



Vengono naturali due problemi (di cui vorremo costruire degli oracoli):
\begin{itemize}
\item restituire, al variare di $i,j$ se $S_i \cap S_j$ contiene elementi;
\item enumerare, al variare di $i,j$, gli elementi di $S_i \cap S_j$.
\end{itemize}

\subsection{Intersezione non vuota}

Consideriamo il problema di dire se l'intersezione di due insiemi \`e
non vuota. Un oracolo semplice per rispondere a questa domanda
consiste nel salvarsi la tabella di $n^2$ bit che risponde a questa
domanda.

Non sappiamo se questo \`e il miglior risultato, per\`o si congettura
che lo sia, vedi per esempio la \textit{Conjecture 3} di
\url{http://u.cs.biu.ac.il/~liamr/papers/focs2010.pdf}.

\subsection{Enumerare gli elementi dell'intersezione}

Noi ci concentreremo su questo problema, vogliamo in particolare
vedere le connessioni con il problema di enumerare tutti i cammini
minimi tra una coppia di nodi di un grafo.

Osserviamo che un oracolo in grado di enumerare le intersezioni $S_i
\cap S_j$ in tempo proporzionale alla loro grandezza \`e anche in
grado di dire in tempo costante se queste intersezioni sono non
vuote. Quindi, per la congettura sopra citata, questo oracolo deve
usare $\Omega (n^2)$ spazio.

Un oracolo banale consiste nel salvare tutte le intersezioni. Vediamo
quanto spazio occupa. Sia $Mem$ questo spazio, possiamo scrivere (con un
conteggio doppio):
\[ 2Mem = \sum _i \sum _j \abs{S_i \cap S_j} \]
osservando che $\abs{S_i \cap S_j} \le \abs{S_i}$ otteniamo
\[ 2Mem \le \sum _i \sum _j \abs{S_i} = \sum _i n \abs{S_i} = nm \]
da cui $Mem = O(nm)$.

\section{Dai cammini alle intersezioni}

Vogliamo costruire un oracolo che enumeri gli elementi di
un'intersezione utilizzando un oracolo per cammini minimi.

Abbiamo quindi un insieme ambiente $X$ e $\Sigma = \set {S_1,... S_n}$
con $S_i \in \mathcal{P}(X)$. Possiamo supporre (a meno di ridurre
$X$) che $x \in X \Leftrightarrow \exists S_i \ni x$.

Costruiamo un grafo indiretto non pesato $G=(V,E)$ con $V = \Sigma
\cup X$ e
\[ E = \set{ (S_i,x) \mid x \in S_i } \]

si vede facilmente che per $i\neq j$ si ha $S_i \cap S_j \neq
\emptyset \Leftrightarrow \delta\pa{S_i,S_j} = 2$. In questo caso
\[ S_i \cap S_j = \set{ x \mid (S_i, x, S_j) \text{ è un cammino
    minimo da $S_i$ a $S_j$} } \]
per cui possiamo enumerare facilmente questo insieme avendo a
disposizione tutti i cammini minimi tra $S_i$ a $S_j$ da cui il nostro
oracolo per intersezioni usando un oracolo per cammini minimi.

Il grafo costruito ha $N = n+ \abs{X}$ nodi e $M = \sum _i \abs{S_i} =
m$ archi.

\section{Dalle intersezioni ai cammini}

Supponiamo di avere un oracolo per enumerare gli elementi di
un'intersezione e costruiamo un oracolo per enumerare tutti i cammini
minimi fra ogni coppia di nodi di un grafo connesso indiretto non
pesato $G=(V,E)$ con $N = \abs{V}$ e $M = \abs{E}$.

Possiamo supporre di aver un oracolo per distanze $\delta$ che occupa
$O(N^2)$ spazio.

Procediamo per induzione sulla distanza fra due nodi.

Se due nodi $s,t$ hanno distanza $0$ allora sono lo stesso nodo e quindi
esiste un unico cammino $(s)$.

Supponiamo di avere un oracolo in grado di restuitire tutti i cammini
minimi tra due nodi di distanza al pi\`u $k-1$ e costruiamo un oracolo
in grado di restutire tutti i cammini minimi fra due nodi $s,t$ di
distanza $k$.

Consideriamo l'insieme $P_{st} = \set{ u \in V \mid \delta \pa{s,u} =
  k-1} \cap \set{ u \in V \mid \delta \pa{u,t} = 1}$.

\begin{mylem}
  Ogni cammino minimo tra $s$ e $t$ \`e nella forma $(s,...,u,t)$ per
  qualche $u\in P_{st}$. Inoltre per ogni $u \in P_{st}$ esiste un
  cammino minimo che assume tale forma.
\end{mylem}
\begin{proof}
  Ogni cammino mimimo di lunghezza $k$ da $s$ a $t$ possiamo scriverlo
  come $(s,..., v,t)$ per un qualche $v\in V$. Ricordando che
  sottocammini di cammini minimi sono ancora minimi otteniamo che il
  cammino (di lunghezza $k-1$) $(s,...,v)$ ha lunghezza $k-1$, inoltre
  deve esistere un arco $(v,t)$, quindi $v\in P_{st}$.

  Se invece $u \in P_{st}$ allora consideriamo un cammino minimo
  $(s,v_1,...v_{k-2},u)$, siccome questo deve avere lunghezza $k-1$
  otteniamo che $(s,v_1,...v_{k-2},u,t)$ \`e un cammino minimo.
\end{proof}

Per ipotesi abbiamo (o meglio: avremo dopo averlo costruito) un
oracolo in grado di enumerarci gli elementi di $P_{st}$. Per ipotesi
induttiva abbiamo anche un oracolo in grado di ritornarci tutti i
cammini minimi tra $s$ e $u \in P_{st}$. Quindi possiamo utilizzare il
seguente algoritmo:

\begin{enumerate}
\item $C = \emptyset$ insieme dei cammini minimi da calcolare;
\item calcola gli elementi di $P_{st}$;
\item $\forall u \in P_{st}$:
  \begin{enumerate}
  \item siano $C_u$ i cammini minimi (di lunghezza $k-1$) tra $s$ e
    $u$;
  \item $C = C \cup \set{ (s,v_1,..,v_{k-2},u,t) \mid
      (s,v_1,...,v_{k-2},u) \in C_u }$;
  \end{enumerate}
\item restituisci $C$.
\end{enumerate}

Si vede facilemente che questo \`e un algoritmo che calcola in tempo
proporzionale all'output tutti i cammini minimi tra $s$ e $t$. Lo
spazio utilizzato \`e uguale allo spazio utilizzato per memorizzare
l'oracolo che ritorna gli insiemi $P_{st}$.

Il modo pi\`u semplice per calcolare gli elementi di $P_{st}$ \`e
utilizzare la definizione:
\[ P_{st} = \set{ u \in V \mid \delta \pa{s,u} = k-1} \cap \set{ u \in
  V \mid \delta \pa{u,t} = 1}\]
consideriamo quindi gli insiemi:
\[ L_s(k) = \set{ u \in V \mid \delta \pa{s,u} = k} \;\;\;\;\;\;
\neigh(t) = L_t(1) \]
in modo da poter scrivere $P_{st} = L_s(\delta\pa{s,t}-1) \cap
\neigh(t)$.

Osserivamo che il massimo $k$ per cui $L_s(k)$ \`e non vuoto per
qualche $s \in V$ \`e il diametro del grafo che chiamiamo $d$.

Esistono quindi $\abs{V} \cdot d = Nd$ insiemi $L_s(k)$ non vuoti al
variare di $s\in V$ e $k \in \mathbb{N}$.

Si ha (usando che $L_s(k) \cap L_s(h) = \emptyset$ per $k\neq h$):
\[ V = \biguplus L_s(k) \Rightarrow \sum _k \abs{L_s(k)} = \abs{V} =
N \]
da cui
\[ \sum _{k,s} \abs{L_s(k)} = \sum _s \sum _k \abs{L_s(k)} = \sum _s N
= N^2 \]

Quindi abbiamo bisogno di un oracolo per intersezioni di insiemi che
lavora su $n = dN$ insiemi la cui somma delle cardinalit\`a \`e $m =
N^2$.

Sappiamo che un tale oracolo deve utilizzare (secondo la congetura)
$\Omega \pa{ n^2} = \Omega\pa{ d^2 N^2}$ spazio, se riuscissimo a
raggiungere tale limite avremmo trovato un buon oracolo per cammini
minimi nel caso di grafi con \textbf{diametro piccolo} (per esempio
nelle reti sociali). In generale, per\`o, $d = O(N)$ per cui questo
oracolo (in ogni caso) utilizza memoria $\Omega\pa{ N^4}$!



\end{document}

