\documentclass{beamer}
%\usetheme{PaloAlto}
%\usetheme{Berlin}
\usetheme{Ilmenau}
\usecolortheme{seahorse}

%\usepackage[utf8]{inputenc}
%\usepackage{default}
%\usepackage[italian]{babel}

%\usepackage{titleref}
%\usepackage{zref-titleref}

\usepackage{amsmath}
\usepackage{amssymb}
\usepackage{amsthm}
\usepackage{xfrac}
\usepackage[all]{xy}
\usepackage{mathtools}
\usepackage{graphicx}
%\usepackage{fullpage}
\usepackage{hyperref}
\usepackage[utf8x]{inputenc}
\usepackage[italian]{babel}
\usepackage{mathtools}
\usepackage{colortbl}
\usepackage{multirow}
\usepackage{algorithm}
\usepackage[noend]{algpseudocode}


%\usepackage{pdftricks}
%\begin{psinputs}
%   \usepackage[pdf]{pstricks}
 %  \usepackage{multido}
%\end{psinputs}

\usepackage{ulem}

\setlength{\parindent}{0in}

\newcounter{counter1}

\theoremstyle{plain}
\newtheorem{myteo}[counter1]{Teorema}
\newtheorem{mylem}[counter1]{Lemma}
\newtheorem{mypro}[counter1]{Proposizione}
\newtheorem{mycor}[counter1]{Corollario}
\newtheorem{mycon}[counter1]{Congettura}
\newtheorem*{myteo*}{Teorema}
\newtheorem*{mylem*}{Lemma}
\newtheorem*{mypro*}{Proposizione}
\newtheorem*{mycor*}{Corollario}
\newtheorem*{mycon*}{Congettura}

\theoremstyle{definition}
\newtheorem{mydef}[counter1]{Definizione}
\newtheorem{myes}[counter1]{Esempio}
\newtheorem{myex}[counter1]{Esercizio}
\newtheorem*{mydef*}{Definizione}
\newtheorem*{myes*}{Esempio}
\newtheorem*{myex*}{Esercizio}

\theoremstyle{remark}
\newtheorem{mynot}[counter1]{Nota}
\newtheorem{myoss}[counter1]{Osservazione}
\newtheorem*{mynot*}{Nota}
\newtheorem*{myoss*}{Osservazione}

\newcommand{\obar}[1]{\overline{#1}}
\newcommand{\ubar}[1]{\underline{#1}}

\newcommand{\set}[1]{\left\{#1\right\}}
\newcommand{\pa}[1]{\left(#1\right)}
\newcommand{\ang}[1]{\left<#1\right>}
\newcommand{\bra}[1]{\left[#1\right]}
\newcommand{\abs}[1]{\left|#1\right|}
\newcommand{\norm}[1]{\left\|#1\right\|}
\newcommand{\ceil}[1]{\left\lceil#1\right\rceil}
\newcommand{\floor}[1]{\left\lfloor#1\right\rfloor}

\newcommand{\pfrac}[2]{\pa{\frac{#1}{#2}}}
\newcommand{\bfrac}[2]{\bra{\frac{#1}{#2}}}
\newcommand{\psfrac}[2]{\pa{\sfrac{#1}{#2}}}
\newcommand{\bsfrac}[2]{\bra{\sfrac{#1}{#2}}}

\newcommand{\der}[2]{\frac{\partial #1}{\partial #2}}
\newcommand{\pder}[2]{\pfrac{\partial #1}{\partial #2}}
\newcommand{\sder}[2]{\sfrac{\partial #1}{\partial #2}}
\newcommand{\psder}[2]{\psfrac{\partial #1}{\partial #2}}

\newcommand{\intl}{\int \limits}

\DeclareMathOperator{\de}{d}
\DeclareMathOperator{\id}{Id}
\DeclareMathOperator{\len}{len}

\DeclareMathOperator{\gl}{GL}
\DeclareMathOperator{\aff}{Aff}
\DeclareMathOperator{\isom}{Isom}

\DeclareMathOperator{\im}{Im}

\DeclareMathOperator{\neigh}{neigh}



\begin{document}


\title[Oracoli per cammini minimi su grafi - bonus]{Cammini minimi su grafi:\\
  oracoli per distanze e cammini}
%\subtitle{}
%\author{Enrico Polesel}
%\institute[Scuola Normale Superiore]{Scuola Normale Superiore}
\date{27 aprile 2015}

\author[Enrico Polesel]{\begin{tabular}{r@{ }l}
Autore: &  Enrico Polesel \\ 
Relatore: & Prof. Roberto Grossi
\end{tabular}
}



%\begin{frame}[plain]
%  \titlepage
%\end{frame}

%\begin{frame}[plain]
% \frametitle{Indice}
% \tableofcontents
%\end{frame}


%\AtBeginSection[]
%{
%  \begin{frame}{\secname}
%    \tableofcontents[currentsection]
%  \end{frame}
%}


\AtBeginSubsection[]
{
  \begin{frame}[plain]{\secname $\rightarrow$ \subsecname}
    \tableofcontents[currentsubsection]
  \end{frame}
}

\section{Contenuti bonus}

\subsection{Grafi planari}

\begin{frame}{Generalizzazione della costruzione dei cammini minimi}
  Scegliamo un  $M \subseteq V_{st}$.

  Al variare di $u \in M$ tutti i cammini ottenuti concatenando un
  cammino minimo da $s$ a $u$ con uno da $u$ a $t$ sono cammini minimi
  da $s$ a $t$.
  \vfill

  Se $M$ \`e scelto in modo opportuno (per esempio $M = P_{st}$)
  allora questi saranno anche i soli cammini minimi da $s$ a $t$.
\end{frame}


\begin{frame}{Separation theorem}
  \begin{myteo}[\textit{Separation theorem}]
    Dato un grafo planare $G = (V,E)$ \`e possibile partizionare $V$
    in tre insiemi $A,B,S$ tali che
    \begin{itemize}
    \item $A$ e $B$ hanno al pi\`u $2N/3$ vertici ciascuno
    \item $S$ ha $O\pa{ \sqrt{N}}$ vertici
    \item non esistono archi tra $A$ e $B$
    \end{itemize}
  \end{myteo}
  \vfill
  \pause
  Sottografi di grafi planari sono planari, quindi possiamo applicare
  ricorsivamente il teorema a $A\cup S$ e $B\cup S$.
\end{frame}

\begin{frame}{Dati da mantenere}
  La struttura data dal \textit{separation theorem} ci permette di
  procedere con una strategia di \textit{divide et impera}.
  \vfill \pause

  Per ogni coppia $s,t\in V$ ci salviamo i nodi separanti ``buoni'',
  cio\`e (detto $S$ un insieme che separa $s$ e $t$ dato dal
  separation theorem):
  \[ S_{st} = \set{ u \mid u \in S \wedge \delta\pa{s,u} +
    \delta\pa{u,t} = \delta\pa{s,t} } \]
  
  questi sono, al pi\`u, $O\pa{\sqrt{N}}$.
  \vfill \pause

  \begin{block}{Spazio utilizzato}
    \[ O \pa{N^{2.5}\log N} \text{ bit} \]
  \end{block}
\end{frame}

\begin{frame}{Un oracolo}
  Supponiamo di esserci salvati i $S_{st}$ al variare di $s,t\in V$.

  Possiamo costruire tutti i cammini minimi fra $s$ e $t$ come la
  congiunzione di tutti i cammini minimi tra $s$ e $u$ e tutti i
  cammini minimi tra $u$ e $t$.
  \vfill

  Questo \`e effettivamente un oracolo e lo spazio utilizzato \`e
  quello per salvare i $S_{st}$ che abbiamo detto essere $O\pa{
    N^{2.5} \log n}$ bit.
\end{frame}


\subsection{APSP}
\label{sec:APSP}

\begin{frame}{APSP}
  Dato un grafo $G=(V,E)$ vogliamo calcolare la tabella delle
  distanze.
  
  \begin{itemize}
  \item Applicare, per ogni sorgente, Dijkstra (archi di peso non
    negativo) o Bellman-Ford: $O(NM + N^2 \log N)$ o $O(N^2M)$ tempo.
  \item Usare Floyd-Warshall: $O(N^3)$ tempo.
  \item (1999) usando il prodotto veloce tra matrici: $\tilde O(kN^\omega)$
    (con archi interi lunghi al pi\`u $k$)
  \item (2014) usando il min-plus product: $\frac{n^3}{2^{\Omega(\log
        n)^{1/2}}}$ (archi interi).
  \end{itemize}
\end{frame}

\begin{frame}{Lower bound di APSP}
  \begin{mycon}
    Esiste una costante $c$ tale che, in un modello RAM con word da
    $O(\log N)$ bit, ogni algoritmo per APSP impiega almeno
    $n^{3-o(1)}$ tempo per costruire la tabella delle distanze per un
    grafo con archi di peso in $\set{ 1,...,n^c}$.
  \end{mycon}
\end{frame}

\begin{frame}{Matrice di adiacenza}
  Numeriamo i vertici: $V = \set{ 0,1,2,..., N-1}$.
  
  \vfill
  Definiamo la matrice $W$ dove
  \begin{itemize}
  \item $W_{i,i} = 0$
  \item $W_{i,j} = w(i,j)$ se $(i,j) \in E$
  \item $W_{i,j} = \infty$ se $(i,j) \not\in E$
  \end{itemize}
 
\end{frame}

\begin{frame}{Calcolo della matrice delle distanze}
  Matrice delle distanze: $d_{i,j} = \delta(i,j)$
  \pause
  \vfill
  
  $l^{(m)}_{i,j} = \min\set{ l^{(m-1)} _{i,j} , \min _{0\le k\le N-1}
    \set{ l^{(m-1)} _{i,k} + w_{k,j}} }$
  \pause
  
  Ogni cammino minimo ha al pi\`u $N-1$ archi $\Rightarrow$ $d_{i,j} =
  l^{(N-1)} _{i,j}$.  \pause \vfill
  
  Possiamo riscrivere $l^{(m)}_{i,j} = \min \limits _{0\le k\le N-1}
  \set{ l^{(m-1)} _{i,k} + w_{k,j}}$
\end{frame}

\begin{frame}{Lettura del prodotto fra matrici}
  \begin{block}{}
    \begin{align*}
      l^{(m)}_{i,j} &=& & \min \limits _k & \Big\{ & l^{(m-1)}
      _{i,k} & & +  & & w_{k,j} \Big\} \\
      C_{i,j} &=& &\sum _k  & &  A_{i,k}  & & \cdot  & &B_{k,j} 
    \end{align*}
  \end{block}
  \pause
  
  \[ L^{(m)} = L^{(m-1)} * W \]

  \begin{block}{}
    \[ D = W^{N-1} \]
  \end{block}
\end{frame}

\begin{frame}{Complessit\`a in tempo}
  \begin{overprint}
    \onslide<1>
    \begin{center}
      \begin{tabular}{| c | c |}
        \hline
        Calcolo di un prodotto & Elevamento a potenza \\
        \hline
        $ A *B $ & $W ^{N-1}$ \\
        $O\pa{N^3}$ & $O\pa{ N}$ \\
        \hline
        \multicolumn{2}{| c| }{$O\pa{N^4}$}\\
        \hline
      \end{tabular}
    \end{center}
    \onslide<2>
    \begin{center}
      \begin{tabular}{| c | c |}
        \hline
        Calcolo di un prodotto & Elevamento a potenza \\
        \hline
        $ A *B $ & \cellcolor{yellow} $W ^{2^{\ceil{\log\pa{N}}}}$ \\
        $O\pa{N^3}$ & \cellcolor{yellow} $O\pa{ \log\pa{N}}$ \\
        \hline
        \multicolumn{2}{| c| }{$O\pa{N^3 \log\pa{N}}$}\\
        \hline
      \end{tabular}
    \end{center}
    \onslide<3->
    \begin{center}
      \begin{tabular}{| c | c |}
        \hline
        Calcolo di un prodotto & Elevamento a potenza \\
        \hline
        $ A *B $ & $W ^{2^{\ceil{\log\pa{N}}}}$ \\
        \color{red} \bf ??? $O\pa{N^\omega}$ ??? & $O\pa{ \log\pa{N}}$ \\
        \hline
        \multicolumn{2}{| c| }{\color{red} ?? $O\pa{N^\omega \log\pa{N}}$ ??}\\
        \hline
      \end{tabular}
    \end{center}
  \end{overprint}
  \vfill
  \begin{overprint}
    \onslide<3-> $O\pa{N^\omega}$ \`e la complessit\`a in tempo del
    prodotto veloce fra matrici.

    Il valore di $2\le\omega <3$ \`e un problema aperto.
  \end{overprint}
  \vfill
  \begin{overprint}
    \onslide<4-> Il prodotto veloce funziona solo su anelli.
    
    $\pa{\mathbb{R},\min, +}$ non \`e un anello!!
  \end{overprint}
  \vfill
  \begin{overprint}
    \onslide<5-> Per archi di peso intero positivo: $O\pa{ d N ^\omega
      \log d \log N}$ {\color{white}\ref{frame:prodottointeri}}
  \end{overprint}
\end{frame}

\begin{frame}{Prodotto $\min + $ fra interi}
\label{frame:prodottointeri}
  \begin{overprint}
    \onslide<1>
    \begin{equation*}
      \overbrace{
        \begin{pmatrix}
          a_{11}  & \cdots & a_{1N} \\
          \vdots & \ddots & \vdots \\
          a_{N1} & \cdots & a_{NN} 
        \end{pmatrix}} ^{A} *
      \overbrace{
        \begin{pmatrix}
          b_{11}  & \cdots & b_{1N} \\
          \vdots & \ddots & \vdots \\
          b_{N1} & \cdots & b_{NN} 
        \end{pmatrix}} ^{B} =
      \overbrace{
        \begin{pmatrix}
          c_{11}  & \cdots & c_{1N} \\
          \vdots & \ddots & \vdots \\
          c_{N1} & \cdots & c_{NN} 
        \end{pmatrix}} ^{C}
    \end{equation*}
    \onslide<2->
    \begin{equation*}
      \overbrace{
        \begin{pmatrix}
          x^{a_{11}}  & \cdots & x^{a_{1N}} \\
          \vdots & \ddots & \vdots \\
          x^{a_{N1}} & \cdots & x^{a_{NN}} 
        \end{pmatrix}} ^{A'} \cdot
      \overbrace{
        \begin{pmatrix}
          x^{b_{11}}  & \cdots & x^{b_{1N}} \\
          \vdots & \ddots & \vdots \\
          x^{b_{N1}} & \cdots & x^{b_{NN}} 
        \end{pmatrix}} ^{B'} =
      \overbrace{
      \begin{pmatrix}
        c'_{11}  & \cdots & c'_{1N} \\
        \vdots & \ddots & \vdots \\
        c'_{N1} & \cdots & c'_{NN} 
      \end{pmatrix}}^{C'}
    \end{equation*}
  \end{overprint}
  \begin{overprint}
    \onslide<2-> \[ c'_{ij} = \sum _k x^{a_{ik} + b_{kj}} \]
  \end{overprint}
  \[ c_{ij} = \min _k \set{ a_{ik} + b_{kj} } \]
  \begin{overprint}
    \onslide<3-> \[ c_{ij} = \mathrm{first}\pa{c'_{ij}} \]
  \end{overprint}
\end{frame}

\subsection{$k$ cammini pi\`u brevi}
\label{sec:kcamm}

\begin{frame}
  \begin{block}{Problema}
    Dati $u,v \in V$ trovare i $k$ cammini pi\`u brevi in $P(u,v)$.
  \end{block}
  \vfill
  
  \begin{itemize}
  \item Alcuni algoritmi venono forniti in \textit{Eppstein 1994}
  \item In \textit{Aljazzar-Leue 2011}, senza peggiorare le
    prestazioni al caso pessimo, si riesce a risolvere il problema
    senza esplorare tutto il grafo
  \end{itemize}
  
\end{frame}

\subsection{Oracoli per distanze esatti}
\label{sec:oracolidist}

\begin{frame}{Albero etichettato}
  Sia $T$ un albero di ricoprimento qualsiasi di $G$ con radice in
  $r$. Denotiamo con $f(u)$ il padre di $u$.
  \vfill
  \[ \delta\pa{ s,u} = \delta\pa{ s,r} + \sum _{v\in \pi (u)} \bra{
    \delta\pa{ s,v} - \delta\pa{ s,f(v) } } \]
  
  Per la triangolare si ha $\abs{\delta\pa{ s,v} - \delta\pa{ s,f(v)}
  } \le \abs{\delta\pa{v,f(v)}} \le 1$ \vfill
  
  \pause
  \[ l_s(v) = \delta\pa{s,v} - \delta\pa{ s, f(v) } \in \set{
    -1,0,1} \]
\end{frame}

\begin{frame}{Somme prefisse}
  \begin{columns}
    \begin{column}{0.6\textwidth}
      \includegraphics[width=\textwidth]{labeltree}
    \end{column}
    \begin{column}{0.4\textwidth}
      Associamo ad ogni etichettatura $l_s$ una sequenza $L_s$.

      Per ogni nodo $u\in V$ nella stringa compare $l(u)$ (triangolino
      verde) e $-l(u)$ (quadratino rosso).  \vfill \pause
      
      $\sum _{v\in \pi(u)} \bra{ \delta\pa{ s,v} - \delta\pa{ s,f(v) } }$
      \`e la somma prefissa fino a $r(u)$.
    \end{column}
  \end{columns}
\end{frame}

\begin{frame}{Complessità in spazio}
  Dobbiamo salvare:
  \begin{itemize}
  \item albero: $O\pa{N}$ bit
  \item distanze $\delta\pa{s,r}$ per ogni $s\in V$: $O\pa{ N \log N}$
    bit totali
  \item vettore $r$: $O\pa{N \log\pa{N}}$ bit 
  \item al variare di $s\in V$ una struttura dati associata a $L_s$:
    $2\log\pa{3} N^2 + o(N^2)$ bit totali {\color{white}
      \ref{frame:wavelettree}}
  \end{itemize}
  Asintoticamente lo spazio utilizzato è dato dalla struttura per
  $L_s$.

  \textbf{Totale:} ${2}\log\pa{3} N^2 + o(N^2)$ bit \vfill
  \pause

  Si pu\`o migliorare a 
  \begin{block}{Bit utilizzati}
    \[ \frac{1}{2}\log\pa{3} N^2 + o\pa{N^2} = 1.585 \frac{N^2}{2} +
    o\pa{N^2} \]
  \end{block}
\end{frame}

\begin{frame}{Wavelet Tree}
  \label{frame:wavelettree}
  \begin{myteo}
    Sia $S \in\Sigma ^n$. Il Wavelet Tree costruito su $S$ occupa
    $nH_0(S) + o(n)$ bit e risponde in $\log \abs{\Sigma}$ tempo a:
    \begin{itemize}
    \item Ottieni il carattere $S[i]$
    \item $Rank_c (S,i)$: il numero di occorenze del carattere~ $c\in
      \Sigma$ nella stringa $S[0:i]$
    \item $Select _c (S,i)$: la posizione dell'$i$-esima occorrenza
      di~ $c\in \Sigma$ in~ $S$
    \end{itemize}
  \end{myteo}

  \[ \sum _{i=0} ^ {k} L[i] = \sum _{\sigma \in \Sigma} \sigma Rank
  _\sigma (S,k) \]
\end{frame}


\subsection{Oracoli per distanze approssimati}
\label{sec:approx}

\begin{frame}{Approssimazioni}
  Dover usare $\Omega \pa{N^2}$ bit potrebbe essere comunque troppo,
  possiamo rinunciare ad avere le distanze precise in cambio di
  un'ulteriore diminuzione dello spazio occupato dall'oracolo.
  \vfill
  
  \begin{mydef}[$r$-approssimazione]
    Un algoritmo che ritorna $d(u,v)$ tale che
    \[ d\pa{u,v} \le r \delta\pa{u,v} \]
  \end{mydef}
\end{frame}

\begin{frame}{}
\setlength{\tabcolsep}{2pt}
  \begin{tabular}{c| c| c|c|c}
    \textbf{autore} & \textbf{preprocessing} & \textbf{spazio} & \textbf{tempo} & \textbf{approx} \\
    \hline 
    \multirow{3}{*}{
    \begin{tabular}{c}
      Thorup-Zwick\\
      STOC 2001
    \end{tabular}}
  & $O\pa{kMN^{1/k}}$ & $O\pa{kN^{1+1/k}}$ &
  $O\pa{k}$ & 2k-1 \\
  & $O\pa{MN^{1/2}}$ & $O\pa{N^{3/2}}$ & $O(1)$ & $3$\\
  & $O\pa{M\log N}$ & $O\pa{N\log{N}}$ & $O\pa{\log N}$  & $\log N$ \\
\hline
     \begin{tabular}{c}
       Baswana-Gaur\\
       Sen-Upadhyay\\
       LNCS 2008
     \end{tabular}
     & 
     \begin{tabular}{c}
       $O\big( M + kN^{3/2}$\\
       $N^{1/(2k)}$\\
       $N^{1/\pa{2k-2}}\big)$
     \end{tabular}
     & $O\pa{kN^{1+1/k}}$ & $O\pa{k}$ & $\pa{2k-1,2}$ 
  \end{tabular}
\setlength{\tabcolsep}{6pt}
\end{frame}

\begin{frame}{Graph spanner}
    \begin{mydef}[$t$-spanner]
    $G' = (V,E')$ con $E' \subseteq E$ tale che la distanza in $G'$
    \`e una $t$-approssimazione.
  \end{mydef}

  \begin{myteo}
    Dato un grafo $G = (V,E)$ e due numeri $t,m\ge 1$ il problema di
    determinare se esiste un $t$-spanner di $G$ con al pi\`u $m$ archi
    \`e NP-completo.
  \end{myteo}
\end{frame}

\subsection{Miglioramento dell'oracolo per cammini}
\label{sec:oracolocamminipiu}

\begin{frame}{Miglioramento}
  \begin{block}{}
    Se $w$ appartiene ad un cammino minimo in $G_{st}$ allora
    costruiamo (ricorsivamente) tutti i cammini minimi in $G_{sw}$ e
    $G_{wt}$. Concatenandoli otteniamo solo (ma non tutti) cammini
    minimi in $G_{st}$.
  \end{block}
  \vfill

  Ricorsivamente, per ogni cammino minimo in $G_{st}$ che
  scopriamo, applichiamo questo ragionamento ai suoi nodi.  
\end{frame}

\begin{frame}{Struttura dei cammini}
  Consideriamo il DAG $G_{st}$ dei cammini da $s$ a $t$ (che \`e
  connesso come grafo indiretto), se togliamo i due estremi otteniamo
  delle componenti connesse $H_1,...,H_k$.

  \begin{mypro}
    Per ogni componente $H_i$ ci basta conoscere un vertice $h_i \in H_i$
    per conoscere tutta la componente
  \end{mypro}
  \vfill \pause

  Invece di tutto l'insieme $P_{st}$ salviamo un elemento per ogni
  componete connessa (che potrebbe contenere più elementi di
  $P_{st}$.
\end{frame}

\begin{frame}{Caso pessimo}
%    \begin{columns}
%    \begin{column}{0.6\textwidth}
      \begin{overprint}
        \onslide<1> \includegraphics[width=\textwidth]{diamante}
        \onslide<2-> \includegraphics[width=\textwidth]{diamantemultiplo}
        \begin{center}
          $\Theta \pa{N^3\log N}$ spazio!
        \end{center}
      \end{overprint}
%    \end{column}
%    \begin{column}{0.39\textwidth}
 %    \end{column}
%  \end{columns}

\end{frame}

\end{document}